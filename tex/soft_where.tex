

\documentclass{article}
\usepackage[utf8]{inputenc}
\usepackage{setspace}
\usepackage{ mathrsfs }
\usepackage{graphicx}
\usepackage{amssymb} %maths
\usepackage{amsmath} %maths
\usepackage[margin=0.2in]{geometry}
\usepackage{graphicx}
\usepackage{ulem}
\setlength{\parindent}{0pt}
\setlength{\parskip}{10pt}
\usepackage{hyperref}
\usepackage[autostyle]{csquotes}
\usepackage{hyperref}

\usepackage{cancel}
\renewcommand{\i}{\textit}
\renewcommand{\b}{\textbf}
\newcommand{\q}{\enquote}
%\vskip1.0in





\begin{document}

{\setstretch{0.0}{

\section*{INTRODUCTION}

I work lately primarily with cellular automata (both continuous and discrete ) and symmetric cryptosystems. I think of the algorithms themselves as art (or perhaps as industrial design), and I often use the algorithms to create art in the more typical sense. This pdf is something like a directory to my current, featured projects. I provide a link to more information and source code at the bottom of each entry.


\section*{Thorium}

This symmetric cryptosystem uses a square matrix of bits as a key. The rows and columns are \q{circular shifted,} so that the matrix is topologically a torus (not exactly but \i{somewhat} like Rubik's Cube.)  \b{Cesium}, just below, includes \b{Thorium} as a special case, but I keep \b{Thorium} separate because I hope to implement it in assembly, and the base 2 version came first.


\url{https://k0ntinuum.github.io/thorium-C/}


\section*{Cesium}

This system is generalization of \b{Thorium} to an arbitrary base.  It's easy to just use base 27 (the alphabet and an underscore), so that one can exchange messages in right away in English. I hope to write up a version in Javascript so that I can provide the encoder on a static website for actual use. I teach math, and I'd like to give my students an example of a complex but still invertible function.  

\url{https://k0ntinuum.github.io/cesium-C/}



}}
\end{document}
